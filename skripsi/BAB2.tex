\documentclass{report}
\begin{document}
\chapter{Tinjauan Pustaka dan Landasan Teori}
\section{Tinjauan Pustaka}
\subsection{Penelitian 1 : Generate Text menggunakan RNN}
Pada penelitian Ilya\cite{sutskever2011generating} telah mendemonstrasikan \textit{Multiplicative} RNN atau MRNN untuk memprediksi teks dengan dataset terdiri dari 86 karakter alfabet dengan ukuran file 100MB termasuk angka dan simbol.
Menghasilkan bahwa MRNN dapat memperdiksi dengan lebih baik dibandingkan dengan \textit{Memoizer} \textit{PAQ}
dimana MRNN dapat memprediksi hasil training wikipedia dengan akurasi sebesar 34\%, sedangkan \textit{Memoizer} memprediksi
hasil training dengan akurasi sebesar 27\%.
\subsection{Penelitian 2 : Meneliti tentang IDS dengan metode Regex}

\subsection{Penelitian 3 : Meneliti tentang RNN untuk bytecode}

\subsection{Penelitian 4 : Meneliti tentang Bagaimana kecerdasan buatan berguna pada keamanan jaringan}

\section{Landasan Teori}
\subsection{Data pada jaringan}
Data pada jaringan berbentuk Frame Ethernet dengan ukuran \textit{Maximum Transmission Unit} (MTU) tiap frame sebesar 1500 bytes. Untuk transmisi data lebih besar dari 1500 bytes akan dilakukan pemecahan per frame.
\subsection{Snort IDS}
\subsubsection{Cara kerja}

\subsubsection{Cara penggunaan}
\subsubsection{Jenis jenis rule yang digunakan}
\subsection{Perl Compatible Regular Expression}
\subsubsection{Finite Automata Regular Expression}
\subsubsection{Standar BRE dan ERE}
\end{document}