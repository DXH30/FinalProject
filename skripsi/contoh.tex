\documentclass[a4paper,12pt]{book}
%==============================================================================================
% Bagian deklarasi paket-paket LaTeX yang dibutuhkan
%==============================================================================================
\usepackage{xcolor}
\usepackage{graphicx}
\usepackage[colorlinks]{hyperref}
\usepackage[indonesian]{babel}

\title{Skripsi}
\usepackage{ctable}
\usepackage{fancyhdr}

\usepackage{floatflt}
\usepackage{listings}
\usepackage{subfig}
\usepackage{caption}
\usepackage{eurosym}
\usepackage{colortbl}
\usepackage{paralist}
\usepackage{fancybox}
\usepackage{multicol}
\usepackage{amsmath}
\usepackage[top=4cm, left=4cm, right=2.5cm, bottom=3.5cm]{geometry} % silakan disesuaikan dengan aturan margin buku skripsi yang diberikan oleh universitas Anda.

%==============================================================================================
% Bagian deklarasi tampilan caption pada gambar atau tabel
%==============================================================================================
\DeclareCaptionStyle{italic}
{labelfont={rm,bf},textfont={rm},indention=18pt,labelsep=period,justification=centering}
\captionsetup[figure]{style=italic}
\captionsetup{style=default,labelfont={rm,bf}, format=hang}
%==============================================================================================
% Bagian input data untuk keterangan file PDF yang dihasilkan
%==============================================================================================
\pdfinfo{
/Author ( ) % silakan diisi dengan nama pembuat skripsi
/CreationDate (D:20060311110000) % D:YYYYMMDDhhmmss (diisi dengan tanggal pembuatan skripsi)
/ModDate (D:20060311110000) % diisi dengan tanggal modifikasi skripsi
/Creator (LaTeX) 
/Producer (pdfLaTeX)
/Title ( ) % diisi dengan nama file skripsi
/Subject ( ) % diisi dengan judul skripsi
/Keywords ( ) % diisi dengan kata-kata kunci isi skripsi
}
%==============================================================================================
% Bagian pengaturan hyperref
%==============================================================================================
\hypersetup{
pdftitle={ }, % diisi dengan judul akhir skripsi
pdfauthor={ }, % diisi dengan nama penulis skripsi
pdfkeywords={ }, % diisi dengan kata-kata kunci dari isi skripsi
bookmarksnumbered,
pdfstartview={FitH},
urlcolor=cyan, % warna URL yang ditampilkan, boleh diganti selain CYAN.
}

%==============================================================================================
% Koleksi pemenggalan kata, isi bagian ini bisa ditambah jika Anda mendapati kata yang salah pemenggalannya
%==============================================================================================
\hyphenation{deng-an}
\hyphenation{lang-uage}

\hyphenation{kon-fi-gu-ra-si}
\hyphenation{me-nang-an-i}
\hyphenation{ter-se-but}
\hyphenation{kom-pu-ter}
\hyphenation{ber-sa-ma-an}
\hyphenation{di-mi-liki}
\hyphenation{pa-sien}
\hyphenation{bia-sa-nya}
%==============================================================================================
% Bagian pengaturan paragraf
%==============================================================================================
\setlength{\parindent}{0in} % jika awal paragraf ingin dibuat menjorok ke dalam, ganti angka 0 nya.
\newcommand{\marginal}[1]{\leavevmode\marginpar{\color{cyan}\footnotesize\raggedright#1\par}}
%==============================================================================================
% BAGIAN ISI DOKUMEN
%==============================================================================================
\begin{document}  
%==============================================================================================
% Bagian pembuatan halaman judul
%==============================================================================================
\pagestyle{empty}
     
    \begin{titlepage}
      \begin{center}
      {\Large\textbf{    }}\\ % diisi dengan judul skripsi
      \par
      \vspace{3cm}
     
      \textbf{\large SKRIPSI}
      \end{center}
        \vspace{1cm}
      \hspace{6,8cm}oleh :\\
      \hspace{5,5cm} Nama Penulis \\ % ganti Nama Penulis dengan nama Anda sendiri
      \hspace{4,3cm}  NIM / Jurusan % ganti NIM dan Jurusan dengan nomor induk mahasiswa dan jurusan tempat Anda kuliah
     
      \begin{center}
      \par   
      \vfill 
        \begin{figure}[h]
          \hspace{6cm} 
          \includegraphics[width=2cm]{strate} % isi dengan nama file logo Universitas Anda
        \end{figure}
        \vspace*{3cm}
      {\bf \large NAMA PROGRAM STUDI}\\ % isi dengan program studi yang Anda ikuti
      {\bf \large NAMA FAKULTAS}\\ % isi dengan nama fakultas tempat Anda kuliah
      {\bf \large NAMA UNIVERSITAS}\\ % isi dengan nama universitas tempat Anda kuliah
      {\bf  \large 2006 } % sesuaikan tahun dengan tahun kelulusan
      \end{center}
\end{titlepage}

%==============================================================================================
% Bagian pembuatan halaman pengesahan skripsi
%==============================================================================================
\thispagestyle{empty}
\addcontentsline{toc}{chapter}{Halaman Pengesahan}
\begin{center}{%\fontfamily{phv}\selectfont
%\vspace*{0.5cm}
\textbf{\large HALAMAN PENGESAHAN\\[1.3cm]
  {\normalsize \textbf{JUDUL SKRIPSI}}}\\[1.5cm]

  \textbf{\large LAPORAN SKRIPSI}
     
      \par
      \vspace{0.8cm}
      {oleh :
      \\
      \large \textbf{NAMA PENULIS}\\
      NIM / Nama Jurusan
     
      }
\vspace{1.5cm}
Telah disetujui dan disahkan untuk persyaratan memperoleh gelar\\[0.2cm]
SARJANA\\[0.2cm]
pada\\[0.2cm]
Nama Program Studi\\[0.2cm]
Nama Jurusan\\[0.2cm]
Nama Universitas\\[2cm]
Bandung, Juni 2006\\[0.2cm] % sesuaikan tempat dan tanggal kelulusan
Telah diperiksa, disetujui, dan disahkan oleh :\\
Pembimbing I\\[2.3cm]
\textbf{\bfseries \underline{Nama Pembimbing}}\\[0.2cm]
NIP Pembimbing\\

}\end{center}
%==============================================================================================
% Bagian pengaturan header dan footer
%==============================================================================================
\pagestyle{fancy}\fancyhead{}
\renewcommand{\chaptermark}[1]{\markboth{#1}{}}
\renewcommand{\sectionmark}[1]{\markright{\thesection\ #1}}
\fancyhf{}
\fancyfoot[R]{\bfseries\thepage}
\fancyfoot[L]{\bfseries Buku Skripsi}
\fancyhead[R]{\bfseries \rightmark}
\renewcommand{\headrulewidth}{1pt}
\renewcommand{\headrule}{{\color{black}%
\hrule width\headwidth height\headrulewidth \vskip-\headrulewidth}}
\renewcommand{\footrulewidth}{1pt}\addtolength{\headheight}{13.6pt}
\fancypagestyle{plain}{
\fancyhead{}
\renewcommand{\headrulewidth}{0pt}
}
%==============================================================================================
% ISI SKRIPSI
%==============================================================================================
\pagenumbering{roman}
\setcounter{tocdepth}{4}
%==============================================================================================
% AWAL KATA PENGANTAR
%==============================================================================================
\chapter*{KATA PENGANTAR}\addcontentsline{toc}{chapter}{Kata Pengantar}

% isi kata pengantar
% isi kata pengantar
% isi kata pengantar
% isi kata pengantar
\vspace*{2cm}
\begin{flushright}
Bandung, Juni 2006\\ % sesuaikan dengan tempat dan tanggal penulisan skripsi
Penulis,\hspace{0.8cm}
\vspace*{1.5cm}
Nama Penulis % ganti dengan nama pembuat skripsi
\end{flushright}
%==============================================================================================
% AKHIR KATA PENGANTAR
%==============================================================================================
\tableofcontents  %perintah untuk menampilkan daftar isi
\addcontentsline{toc}{chapter}{Daftar Isi}
\listoftables %perintah untuk menampilkan daftar tabel
\addcontentsline{toc}{chapter}{Daftar Tabel}
\listoffigures %perintah untuk menampilkan daftar gambar
\addcontentsline{toc}{chapter}{Daftar Gambar}
%==============================================================================================
% MULAI BAB I
%==============================================================================================
\chapter{PENDAHULUAN} %JUDUL BAB
\label{pendahuluan}
\pagenumbering{arabic}

%==============================================================================================
% Subbab latar belakang
%==============================================================================================
\section{Latar Belakang} %Judul subbab
\label{latarbelakang} 
%isi subbab latar belakang
    \subsection{judul sub subbab}
      %isi sub subbab
      %isi sub subbab
      %isi sub subbab
     
%isi subbab latar belakang
%isi subbab latar belakang
% ISI DARI BAB I INI DAPAT DIKEMBANGKAN TERUS DAN DILANJUTKAN DENGAN BAB-BAB BERIKUTNYA DENGAN
FORMAT YANG SAMA,
% UNTUK MENGHEMAT TEMPAT DALAM CONTOH INI, HANYA DITUNJUKKAN 1 BAB SAJA SEBAGAI CONTOH
%==============================================================================================
% AKHIR BAB I
%==============================================================================================
%==============================================================================================
% Bagian daftar isi
%==============================================================================================
\bibliographystyle{ieeetr}
\bibliography{daftarpustaka}%memasukkan file BibTeX dengan nama "daftarpustaka" yang berisi
kumpulan referensi
\addcontentsline{toc}{chapter}{Daftar Pustaka}

%==============================================================================================
% MULAI BAB LAMPIRAN
%==============================================================================================
\appendix % dengan menggunakan appendix nomor bab akan ditampilan dengan menggunakan huruf;
menjadi Lampiran A

\chapter{Data Pendukung} % contoh ini akan menampilkan bab baru yaitu Lampiran A - Data
Pendukung
%==============================================================================================
% AKHIR DOKUMEN
%==============================================================================================
\newpage
\pagestyle{empty}
\vspace*{7cm}
\centering
\resizebox{7cm}{1cm}{\textrm{\textbf{SELESAI}}}
  \par   
      \vfill 
     
\end{document}